\documentclass{article}

% Add all packages here
\usepackage[utf8]{inputenc}
\usepackage{amsmath, amsthm, amsfonts, amssymb}
\usepackage[margin=1in]{geometry}
\usepackage[backend=biber,sorting=none]{biblatex}
\usepackage[ngerman]{babel}
\addbibresource{ref.bib}

% Useful custom commands I like. Feel free to send a PR :=)

% all number domains
\newcommand{\N}{\mathbb{N}}
\newcommand{\Z}{\mathbb{Z}}
\newcommand{\R}{\mathbb{R}}
\newcommand{\C}{\mathbb{C}}

% short versions of math arrows
\newcommand{\larr}{\leftarrow}
\newcommand{\rarr}{\rightarrow}
\newcommand{\lrarr}{\leftrightarrow}
\newcommand{\Larr}{\Leftarrow}
\newcommand{\Rarr}{\Rightarrow}
\newcommand{\Lrarr}{\Leftrightarrow}

% You can also just overshadow them by overwriting in the main.tex file
\title{Overwrite Me!!!}
\author{Your Name}
\date{\today}


\begin{document}
\maketitle
% ABSTRACT
\section*{Abstract}
Lorem ipsum dolor sit amet, consetetur sadipscing elitr, sed diam nonumy eirmod tempor invidunt ut labore et dolore magna aliquyam erat, sed diam voluptua. At vero eos et accusam et justo duo dolores et ea rebum. Stet clita kasd gubergren, no sea takimata sanctus est Lorem ipsum dolor sit amet. Lorem ipsum dolor sit amet, consetetur sadipscing elitr, sed diam nonumy eirmod tempor invidunt ut labore et dolore magna aliquyam erat, sed diam voluptua. At vero eos et accusam et justo duo dolores et ea rebum. Stet clita kasd gubergren, no sea takimata sanctus est Lorem ipsum dolor sit amet.
\tableofcontents
\newpage


\section{Einführung}
test! \cite{example}
\subsection{Motivation}
\subsection{Ziele und Beitr\"age}
\subsection{Struktur}
\section{Softwarekonzeptionierung und Methodologie}
\subsection{Entwurf einer Markdownerweiterung}
% - Welche möglichen Markdownstandards
%   - Wir basieren auf CommonMark
%   - Aber Unsere targets => GFM, GLFM, HFM
%     - Somit bleibt unsere Syntax damit kompatibel
% - Verschiedene Ansätze wie Tabellen auch aufnehmen
%   - Bilder wie es gerendered wurde
% - Comparison Matrix
\subsection{Unterst\"utzung von \LaTeX-Formeln}
\subsection{Commitbasierte Kartengeneration via Git Hooks}
\section{Implementation}
\subsection{Offlineunterst\"utzung}
\subsection{\LaTeX-Formel Support}
\section{Diskussion}
\subsection{Audio- und Videosupport}
\subsection{Geplante Weiterentwicklung}
\section{Literaturverzeichnis}
\printbibliography
\end{document}
